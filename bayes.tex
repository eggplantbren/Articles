\documentclass[a4paper, 12pt]{article}
\usepackage{graphicx}
\usepackage[numbers]{natbib}
\usepackage{amsmath}
\usepackage{parskip}
\usepackage{dsfont}
\usepackage[left=3cm,top=3cm,right=3cm]{geometry}

\renewcommand{\topfraction}{0.85}
\renewcommand{\textfraction}{0.1}
\parindent=0cm
\newcommand{\btheta}{\boldsymbol{\theta}}
\newcommand{\E}{\mathds{E}}



\title{}
\author{Brendon J. Brewer}

\begin{document}
\sffamily
\maketitle

What is probability? This sounds like a discussion question for a
philosophy class, one of those questions it's fun to think about but which
doesn't have many practical consequences. Surprisingly, this is not the case.
It turns out that different answers to this question lead to completely
different views of how to do statistics and data analysis {\em in practice}.
In the early 20th century this led to a split in the field of statistics,
with intense debates taking place about whose methods and ways of thinking
were better. Unfortunately, the wrong side won the debate and their
ideas still dominate mainstream statistics, a situation which
has exacerbated the reproducibility crises affecting science today.

Here's a common, standard statistical inference problem. An old drug
successfully treats 70\% of patients. To test a new drug, researchers give it
to 100 patients, of whom 76 recover. Based on this evidence, how certain
should we be that the new drug is worse than, identical to, or better than the
old one?


The standard p-value threshold is 0.05. This simply {\em feels} more
convincing than 

Things have improved markedly for Bayesian statisticians over the last few
decades. These days, you're unlikely to encounter any hostility by doing
a Bayesian analysis. By far the dominant attitude these days
is pragmatism. It's possible to do interesting and useful analyses using tools
arising from frequentist thinking, Bayesian thinking, creative invention, or
a mixture of all of these. Most statisticians are happy to do so.
One downside of this ecumenicalism is a reluctance to ask fundamental
questions. Having a strong opinion on this matter has gone out of fashion.
Who's to say one statistical philosophy is better than
another? Aren't all statistical philosophies {\it equally valid} paths to good
data analysis? Frequentism is ``true for me''. As in religion, so in statistics.
If you criticise a colleague for using p-values when posterior probabilities
are clearly more appropriate will lead to accusations of being a `zealot' who
should stop `crusading' \citep{simply_statistics}.

\begin{thebibliography}{999} % if there are less than 10 entries, enter a one digit number
\bibitem[Simply Statistics Blog(2013)]{simply_statistics}
http://simplystatistics.org/2013/11/26/statistical-zealots/

\end{thebibliography}


\end{document}

