\documentclass[a4paper, 12pt]{article}
\usepackage{graphicx}
\usepackage{natbib}
\usepackage{amsmath}
\usepackage{parskip}
\usepackage{dsfont}
\usepackage[left=3cm,top=3cm,right=3cm]{geometry}

\renewcommand{\topfraction}{0.85}
\renewcommand{\textfraction}{0.1}
\parindent=0cm
\newcommand{\btheta}{\boldsymbol{\theta}}
\newcommand{\E}{\mathds{E}}



\title{}
\author{Brendon J. Brewer}

\begin{document}
\sffamily
\maketitle

What is probability? This sounds like a discussion question for a
philosophy class, one of those questions it's fun to think about but which
doesn't have many practical consequences. Surprisingly, this is not the case.
It turns out that different answers to this question lead to completely
different views of how to do statistics and data analysis {\em in practice}.
In the early 20th century this led to a split in the field of statistics,
with intense debates taking place about whose methods and ways of thinking
were better. Unfortunately, the wrong side won the debate and their
ideas still dominate mainstream statistics, a situation which
has exacerbated the reproducibility crises affecting science today.



\end{document}

