\documentclass[a4paper, 11pt]{article}
\usepackage{graphicx}
\usepackage{natbib}
\usepackage{amsmath}
\usepackage{dsfont}
\usepackage[left=3cm,top=3cm,right=3cm]{geometry}

\renewcommand{\topfraction}{0.85}
\renewcommand{\textfraction}{0.1}
\parindent=0cm
\newcommand{\btheta}{\boldsymbol{\theta}}
\newcommand{\E}{\mathds{E}}

\title{The responsibility of statistics in scientific reproducibility}
\author{Brendon J. Brewer}

\begin{document}
\maketitle

It is well known that public opinion can be pushed around by the latest study
finding that some popular food is ``linked'' to an unwanted medical condition,
or that a previously ``unhealthy'' substance, such as butter, is now
``good for you''.

The disciplines of statistics and data science
(my favourite definition of which is {\it statistics done in San Francisco})
are broad and covers many different techniques. For example, some researchers
work on developing practical software tools for visualisation of large and
complicated data sets, while others try to prove theorems about the mathematics
of probability. An important 


Statisticians are perceived by the wider science community as authorities in
quantifying the strength of evidence, among various other skills.


\end{document}

