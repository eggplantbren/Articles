\documentclass[a4paper, 12pt]{article}
\usepackage{graphicx}
\usepackage[numbers]{natbib}
\usepackage{amsmath}
\usepackage{parskip}
\usepackage{dsfont}
\usepackage[left=3cm,top=3cm,right=3cm]{geometry}

\renewcommand{\topfraction}{0.85}
\renewcommand{\textfraction}{0.1}
\parindent=0cm
\newcommand{\btheta}{\boldsymbol{\theta}}
\newcommand{\E}{\mathds{E}}

\title{}
\author{Brendon J. Brewer}

\begin{document}
\sffamily
\maketitle

What is probability? This sounds like a discussion question for a
philosophy class, one of those questions that's fun to think about but that
doesn't have many practical consequences. Surprisingly, this is not the case.
As it turns out, different answers to this question lead to completely
different views of how to do statistics and data analysis {\em in practice}.
In the early 20th century, this led to a split in the field of statistics,
with intense debates taking place about whose methods and ways of thinking
were better. Unfortunately, the wrong side won the debate and their
ideas still dominate mainstream statistics, a situation which
has exacerbated the reproducibility crises affecting science today
\citep{reproducibility}.

Here's a common, standard statistical inference problem. An old drug
successfully treats 70\% of patients. To test a new drug, researchers give it
to 100 patients, 83 of whom recover. Based on this evidence, how certain
should we be that the new drug is worse than, identical to, or better than the
old one? If you think it is legitimate to use the mathematics of probability
theory
to study the concept of {\em plausibility}, you are a `Bayesian' (after Reverend
Thomas Bayes, one of the first people to use probability this way). Faced with
this question, your job is to use probability theory to {\em calculate how plausible it is} that the new
drug is much worse, slightly worse, identical to, slightly better, or much better
than the old one, taking into account the result of the experiment. The answer
you get
depends on both the experimental result itself, as well as what you assume
about how plausible all of those hypotheses were {\em before you knew the result
of the experiment} --- the dreaded `prior'.
Under standard assumptions the `posterior' probability the new drug is better
than the old one is 0.89, the probability it's the same is 0.11 (starting
from a prior of 0.5), and the
probability it's worse is practically zero.

To a `frequentist', there's only one legitimate application of probabilities to
the real world: to relative frequencies or proportions
(plausibility is off
limits because it's too vague to be subject to concrete mathematical rules).
Think of the joke
``84\% of statistics are made up''. It is true that probability theory works
here. But can we apply this to the drug example? We can, but it requires
some imagination.
The standard method is to imagine repeating the experiment many times.
If we did this, the number of patients who recovered would
not be 83 every single time. Sometimes we'd just happen to get more patients
who recover (for whatever reason) and sometimes we'd get less. Interestingly,
by `repeating the experiment' we actually mean `repeating a similar experiment'
--- nobody has the power to repeat anything down to every detail.
With this concept of repetitions, we can talk about {\em what fraction of the
time} a certain experimental result would be observed. If our observed result
is unusual under the assumption of some hypothesis, we have evidence against
that hypothesis. We just need to figure out how to quantify how unusual our
result is.

For example, consider the `null' hypothesis that the new drug
has exactly the same effectiveness as the old drug. Then, under standard assumptions we'd
typically observe about 70 or so of the patients recovering. The fraction of
the time we'd observe 83 recoveries (our actual result) is 0.0012. This number is pretty small, and in
general, any actual experimental result is so specific it would hardly
ever be observed again. Therefore, 0.0012 can't be a good way of quantifying how unusual
our result is. To ``solve'' this problem, the influential mathematical
biologist R. A. Fisher suggested calculating the fraction of the time we would
see our exact result {\em or one that's even more different from what the null
hypothesis would predict}. In our case, we need the fraction of the time we'd
see 83 or more, or 57 or less recoveries, which is 0.006.
This quantity is called a p-value.
An oft-criticised convention is that a p-value less than 0.05 implies evidence
against the null hypothesis, and our result of 0.006 certainly qualifies
as such.

Look at the numbers produced by the two different methods. The frequentist p-value of 0.006 is much lower than the probability the new drug is the same, 0.11.
The two quantities are defined and calculated differently, so {\em of course} they're different. But look how big the difference is.
Doesn't 0.006 just {\em feel}
more compelling? It's 6 out of 1000 --- so close to zero!
The Bayesian probability of 0.11 (that the drugs are the same) is a much
more moderate number. It's quite common for the frequentist's p-value to
be much closer to zero than the Bayesian's posterior probability.
So any time you see a scientific paper claiming a result with $p < 0.05$, this
is {\em nothing like} 95\% certainty that the result is real. This is one
reason why so many experimental results aren't being reproduced; the evidence
for them was never strong anyway. In fact many studies reporting $p < 0.05$
may actually have found evidence {\em in favour} of the null hypothesis, meaning
a ``failed'' replication attempt is actually a success!
Good statisticians know that p-values and posterior probabilities are not the
same thing, and try to impress it upon their students and
scientific colleagues, often without much success.
However, a simpler solution exists. If we want to know
the plausibility of a hypothesis, why don't we just calculate it? If you want
to know $X$, calculate $X$. Don't calculate $Y$ instead, try to use it for
insight about $X$, and then complain
about all the people interpreting $Y$ as if it were $X$. It's not their fault.

The main criticism of Bayesian methods is that they're subjective; the results
depend not just on the data being analysed but also on the other ingredients
you must supply, the prior probabilities (how plausible were your hypotheses
before taking into account the data?). It's true these extra assumptions are
needed, but this is just accepting reality. For example, if we had assumed
the new drug's effectiveness is probably extremely close to that of the old drug,
the data would have been uninformative about the question of ``equal or not'',
and the posterior probability would have equalled the prior probability. This
is not a problem. It is logic.

Things have improved markedly for Bayesian statisticians over the last few
decades. These days, you're unlikely to encounter any hostility by doing
a Bayesian analysis. Pragmatism is the most popular attitude.
It's possible to do interesting and useful analyses using tools
arising from frequentist thinking, Bayesian thinking, creative invention, or
a mixture of all of these, and most statisticians are happy to do so.
One downside of this ecumenicalism is a reluctance to ask fundamental
questions: having a strong opinion on this matter has gone out of fashion.
Who's to say one statistical philosophy is better than
another? Aren't all statistical philosophies {\it equally valid} paths to good
data analysis? Frequentism is ``true for me''. As in religion, so in statistics.
If you criticise a colleague for using p-values when posterior probabilities
are clearly more appropriate, it will lead to accusations of being a `zealot'
\citep{simply_statistics} who should stop `crusading'.
A year ago I went to a talk by prominent `skeptic' Steven Novella, in which
he advocated for a Bayesian approach to judging the plausibility of medical
hypotheses. During the question and answer session, a statistics department colleague of mine raised his hand and said Bayesian statistics was `bullshit' because degrees of plausibility are not empirically measurable. I disagree
strongly, but it was refreshing to see someone willing to argue for a view.

Another, more consequential downside is a reluctance to abandon bad ideas.
Frequentist confidence intervals and p-values should still be taught to some
extent; since so much research is based on them, our students need to know what
they are. Other ideas that arose from frequentism, such as maximum likelihood
estimation and the notion of `bias', are still useful for obtaining fast
approximate results and for computational algorithms respectively.
In some difficult research problems they might help if a Bayesian solution is too mathematically or computationally difficult to obtain.
Yet it is simply false to claim that we teach these methods because they're
better or easier than Bayesian ones. I would argue they are both worse and
more difficult. In physics, undergraduate students learn Newton's ideas about
gravity before Einstein's, because they're much easier conceptually and
mathematically, and give the right answer on many problems even though
Einstein's theory is more correct. If Einstein's theory
were easier (as well as being more accurate), teaching Newton's would be silly. Yet
that's the way most statistics curricula are structured.
The only reason
statisticians think frequentist ideas are easier is that they are used to them.
The only reason they think Bayesian ideas must wait until graduate school is
that the lower-level textbooks haven't been written (although some easier material
exists, such as the books by Allen Downey \citep{downey} and John Kruschke
\citep{kruschke},
and my lecture
notes \citep{331}). It's time to change this. Let's teach Bayes first.

\begin{thebibliography}{999} % if there are less than 10 entries, enter a one digit number
\bibitem[]{reproducibility}
http://www.nature.com/news/over-half-of-psychology-studies-fail-reproducibility-test-1.18248

\bibitem[Simply Statistics Blog(2013)]{simply_statistics}
http://simplystatistics.org/2013/11/26/statistical-zealots/

\bibitem[]{downey}
http://greenteapress.com/thinkbayes/

\bibitem[]{kruschke}
Kruschke, John. Doing Bayesian data analysis: A tutorial introduction with R. Academic Press, 2010.

\bibitem[]{331}
http://github.com/eggplantbren/STATS331

\end{thebibliography}


\end{document}

