\documentclass[a4paper, 11pt]{article}
\usepackage{graphicx}
\usepackage{natbib}
\usepackage{amsmath}
\usepackage{dsfont}
\usepackage[left=3cm,top=3cm,right=3cm]{geometry}

\renewcommand{\topfraction}{0.85}
\renewcommand{\textfraction}{0.1}
\parindent=0cm
\newcommand{\btheta}{\boldsymbol{\theta}}
\newcommand{\E}{\mathds{E}}

\title{The responsibility of statistics in scientific reproducibility}
\author{Brendon J. Brewer}

\begin{document}
\maketitle

It is well known that public opinion can be pushed around by the latest study
finding that some popular food is ``linked'' to an unwanted medical condition,
or that a previously ``unhealthy'' substance, such as butter, is now
``good for you''.

The disciplines of statistics and data science
(my favourite definition of which is {\it statistics done in San Francisco})
are broad and have many subfields. For example, some researchers
work on developing practical software tools for analysing and visualising large and
complicated data sets, while others try to prove theorems about the mathematics
of probability. Many of us also collaborate with applied scientists and find
that even applying standard, well established statistical methods to
real data can become a challenging research problem in itself. In this sense,
statistics and data science will always be a broad church.

However, one fundamental topic which has always been a large part of
statistics is {\it inference}, which is concerned with trying to
draw conclusions from data without fooling yourself. This is a noble goal,
and while it is impossible to be perfect at it, we should try to do the best
we can. Because of this,
statisticians are perceived by the wider science community as authorities
in quantifying the strength of evidence. If a new medical treatments yields 60
recoveries out of 100 patients in a clinical trial, yet 45 out of 100 patients
in the placebo group recover, how certain should we be that the drug really works?

The problem is that most statisticians are wrong about how to do inference.


\end{document}

