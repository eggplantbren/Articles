\documentclass[a4paper, 12pt]{article}
\usepackage{graphicx}
\usepackage[numbers]{natbib}
\usepackage{amsmath}
\usepackage{parskip}
\usepackage{dsfont}
\usepackage[left=3cm,top=3cm,right=3cm]{geometry}

\renewcommand{\topfraction}{0.85}
\renewcommand{\textfraction}{0.1}
\parindent=0cm

\title{}
\author{Brendon J. Brewer}

\begin{document}
\sffamily
\maketitle

The second law of thermodynamics has got to qualify as one of the most
misunderstood principles in all of physics. Depending on who you ask, it is
either incredibly mysterious or entirely mundane. It might be connected to
fundamental ideas such as time and the origin of the universe
\citep{carroll}, and also everyday experiences such as the cooling of a
cup of coffee, or the fact that a room gets messier, rather than tidier,
if you don't clean it up. Yet it can be frustratingly difficult to find a
clear explanation of what it actually is.

I first encountered the second law
as a teenager, while reading a copy of {\em Creation} magazine given to me
by my grandmother. Being a creationist magazine, it defined the second law as
the idea that {\em disorder always increases with time}.
At first glance, this seems incompatible with evolution by
natural selection, which can lead to more complex,
``better designed'' organisms over time \citep{dawkins}.
It seemed unlikely that mainstream biology would flagrantly
contradict physics, so I remained sceptical of this argument,
even though I couldn't understand the explanations I found on the
internet at the time.

During my first undergraduate physics course, I was excited when we came to
study thermodynamics. Finally, I thought, I would be able to understand the
technical details of this fascinating and widely applicable idea.
Instead, we
worked out the maximum possible efficiency of refrigerators and steam engines.
The version of the second law we studied was related to concepts of heat
and temperature. The fact that heat always flows from a hot object to a cooler
one (and not the other way around) is a major everyday example. Your morning
cup of coffee cools down, and heats up the air around it: it doesn't heat
up further while cooling the room, even though that possibility is compatible
with other laws of physics such as the conservation of energy.

This obviously has its uses, but it was virtually impossible to see how this
related to 


Why is golf difficult?
Why can't most men sing operatic high Cs?
Why are three-point shots harder to `hit' in basketball than two-pointers?

A similar, related question is that of how natural selection
puts information into the genome. Reasonable responses include those by
\citep{dawkins2} and \citep{mackay}.

The second law has also been invoked in arguments about the
sustainability of industrial civilization \citep{muse}. Attenborough






\section*{Entropy increases because it stays constant}

\section*{Entropy: it's not what's happening, it's what you know}

\section*{The general principle}
Is there

During my PhD studies, I came across a wonderful paper by the late physicist
E. T. Jaynes, which might be my favourite physics paper of all time
\citep{jaynes}. As with all my favourite papers, I didn't understand it
immediately, but had a strong feeling that I should persist because it seemed
important and profound. Every so often, I'd return to re-read it, understanding
just a little more each time. Then, the breakthrough came after dinner one
night. I took out a tub of ice-cream from the freezer for dessert. After dishing
the ice-cream into a bowl, I tried putting the tub back in the freezer, but
it wouldn't fit. Suddenly it clicked: the second law of thermodynamics is the
principle that {\em big things cannot fit into small spaces, unless you can
compress them}.

I must admit that when I hear `entropy' from a physicist, without a long
set of qualifications, I don't know what they're talking about.
I don't think this is my fault.

There seems to be little recognition of the fact
that the {\em exact same physical system}
can have more than one entropy, depending on how we want to think about it.
What {\em rules of thumb} are we looking for? Different ones may exist, and
we'll find them by using different entropies.

\begin{thebibliography}{999} % if there are less than 10 entries, enter a one digit number
\bibitem[]{carroll}
Carroll, Sean. From eternity to here: the quest for the ultimate theory of time. Penguin, 2010.

\bibitem[]{dawkins}
Dawkins, Richard. The blind watchmaker: Why the evidence of evolution reveals a universe without design. WW Norton \& Company, 1986.

\bibitem[]{dawkins2}
Dawkins, Richard. The information challenge.
http://www.skeptics.com.au/resources/articles/the-information-challenge/

\bibitem[]{jaynes}
Jaynes, Edwin T. Gibbs vs Boltzmann entropies. American Journal of Physics 33.5 (1965): 391-398.

\bibitem[]{mackay}
David MacKay, Information Theory, Inference, and Learning Algorithms.
Chapter 19.
http://www.inference.phy.cam.ac.uk/mackay/itprnn/ps/265.280.pdf

\bibitem[]{muse}
Muse. The 2nd law.
http://www.allmusic.com/album/the-2nd-law-mw0002406726
\end{thebibliography}

\end{document}

