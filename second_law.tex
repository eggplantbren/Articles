\documentclass[a4paper, 12pt]{article}
\usepackage{graphicx}
\usepackage{natbib}
\usepackage{amsmath}
\usepackage{parskip}
\usepackage{dsfont}
\usepackage[left=3cm,top=3cm,right=3cm]{geometry}

\renewcommand{\topfraction}{0.85}
\renewcommand{\textfraction}{0.1}
\parindent=0cm
\newcommand{\btheta}{\boldsymbol{\theta}}
\newcommand{\E}{\mathds{E}}



\title{}
\author{Brendon J. Brewer}

\begin{document}
\sffamily
\maketitle

The second law of thermodynamics has got to qualify as one of the most
misunderstood principles in all of physics.
I first encountered it
as a teenager, while reading a copy of {\em Creation} magazine given to me
by my grandmother. Being a creationist magazine, it defined the second law as
the idea that {\em disorder always increases over time}.
At first glance, this seems incompatible with evolution by
natural selection, which leads to more well-adapted organisms over time.

\end{document}

