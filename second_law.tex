\documentclass[a4paper, 12pt]{article}
\usepackage{graphicx}
\usepackage[numbers]{natbib}
\usepackage{amsmath}
\usepackage{parskip}
\usepackage{dsfont}
\usepackage[left=3cm,top=3cm,right=3cm]{geometry}

\renewcommand{\topfraction}{0.85}
\renewcommand{\textfraction}{0.1}
\parindent=0cm

\title{}
\author{Brendon J. Brewer}

\begin{document}
\sffamily
\maketitle

The second law of thermodynamics has got to qualify as one of the most
misunderstood principles in all of physics. Depending on who you ask, it is
either incredibly mysterious or entirely mundane. It might be connected to
fundamental ideas such as time and the origin of the universe
\citep{carroll}. Or maybe not.

I first encountered the second law
as a teenager, while reading a copy of {\em Creation} magazine given to me
by my grandmother. Being a creationist magazine, it defined the second law as
the idea that {\em disorder always increases with time}.
At first glance, this seems incompatible with evolution by
natural selection, which can lead to more complex,
``better designed'' organisms over time \citep{dawkins}.

\section*{Entropy increases because it stays constant}

\section*{Entropy: it's not what's happening, it's what you know}

\section*{The general principle}
Is there

During my PhD studies, I came across a wonderful paper by the late physicist
E. T. Jaynes, which is probably my favourite physics paper of all time
\citep{jaynes}.

\begin{thebibliography}{999} % if there are less than 10 entries, enter a one digit number
\bibitem[]{carroll}
Carroll, Sean. From eternity to here: the quest for the ultimate theory of time. Penguin, 2010.

\bibitem[]{dawkins}
Dawkins, Richard. The blind watchmaker: Why the evidence of evolution reveals a universe without design. WW Norton \& Company, 1986.

\bibitem[]{jaynes}
Jaynes, Edwin T. Gibbs vs Boltzmann entropies. American Journal of Physics 33.5 (1965): 391-398.
\end{thebibliography}

\end{document}

