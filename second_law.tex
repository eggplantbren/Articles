\documentclass[a4paper, 12pt]{article}
\usepackage{graphicx}
\usepackage[numbers]{natbib}
\usepackage{amsmath}
\usepackage{parskip}
\usepackage{dsfont}
\usepackage{sectsty}
\usepackage[left=3cm,top=3cm,right=3cm]{geometry}

\renewcommand{\topfraction}{0.85}
\renewcommand{\textfraction}{0.1}
\allsectionsfont{\normalfont\sffamily}
\parindent=0cm

\title{Unscrambling the second law}
\author{Brendon J. Brewer}

\begin{document}
\sffamily
\maketitle

\section*{Heat}

The second law of thermodynamics surely qualifies as one of the most
talked-about principles in all of physics. Depending on who you ask, it is
either incredibly mysterious or fairly mundane. Some physicists think
the second law is connected to
fundamental ideas such as time and the origin of the universe
\citep{carroll}. Yet it is also an aspect of everyday experiences
such as the way a morning cup of coffee cools down,
or the fact that you cannot unscramble an egg.
The second law has even been invoked by rock bands to
explain why, in their
view, economic growth cannot continue for much longer \citep{muse}.
Yet trying to find a clear explanation of what the second law actually is
(and why it is true) can be a frustrating experience.

I first encountered the second law
as a teenager, while reading an issue of the fundamentalist Christian magazine
{\em Creation} given to me by my grandmother. Since the magazine wanted to
argue against biological evolution, it claimed that the second law of
thermodynamics does not allow evolution to occur. Its definition of the second
law was that {\em disorder always increases with time}.
At first glance, this does seem incompatible with evolution by
natural selection, which can lead to more complex,
``better designed'' organisms over time \citep{dawkins}.
I always thought it was unlikely that mainstream biology would flagrantly
contradict mainstream physics, so I remained sceptical of this argument,
even though I couldn't understand the explanations I found on the
internet at the time.

During my first undergraduate physics course, I was excited when we came to
study thermodynamics. Finally, I thought, I would be able to understand the
technical details of this fascinating and profound idea.
Alas, my expectations were not met, despite having a really good lecturer.
Instead of discussing big picture issues, we
worked out the maximum possible efficiency of refrigerators and steam engines
(which are probability interesting in their own ways, just not to me).
The version of the second law we studied was related to concepts of heat
and temperature, and little else.
The fact that heat always flows from a hot object to a cooler
one (and not the other way around) is a major everyday example. Your morning
cup of coffee cools down, and heats up the air around it: it doesn't heat
up further while cooling the room, even though that possibility is compatible
with other laws of physics such as the conservation of energy.

In thermodynamics, the second law is formalised by defining a quantity called
{\em entropy}. When heat flows {\em out of} an object and into another,
the first object's entropy goes
down, by an amount that depends on its temperature.
The change in the entropy is the amount of heat energy transferred,
$Q$ (usually measured in joules)
divided by the temperature of the object, $T_1$, measured in Kelvin.
When that same heat energy flows {\em into} another object, that object's
entropy goes up by $Q$ divided by the temperature of the
second object, $T_2$. The second law of thermodynamics can then be stated: if you
add up all of the changes in entropy of all the objects you are studying,
the result must be a positive number or zero. It can't be negative. In other
words, the total entropy must either increase or stay the same.
When a cup of coffee cools down its entropy decreases, but the entropy of its
surroundings increases by an even greater amount,
since the coffee is hotter than the surroundings.

The version of
the second law I just described, usually attributed to the 19th century
German physicist Rudolf Clausius, certainly has its uses. However it is a far
cry from the lofty fundamental principle I had expected to learn. What did it
have to do with evolution? The fact that organisms seem designed doesn't have
anything to do with heat being transferred. And it doesn't have much to do
with the economy either, except very indirectly because machines are useful
and the second law says some kinds of machines aren't possible.

So {\em is there} a version of the second law that relates to concepts more
general than heat and temperature?
It turns out the answer is yes, but I had to wait many years before I learned it.
Surprisingly, it turns out this more general second law isn't really a
principle of {\em physics}, but rather a
principle of {\em honest reasoning}. And this more general version of the
second law explains not only why the Clausius version is true, but gives us
a tool for much more general questions --- like the evolution question.
It also appears in everyday life, and not just in situations involving heat and
temperature. For example,
why is golf difficult? Why can't most men sing operatic high Cs?
And why are political polls (somewhat) accurate?

\section*{Uncertainty, volume, and rules of thumb}
During my PhD studies, I became intensely interested in Bayesian statistics
and how to use it in astronomical data analysis. During this process I
discovered the work of the heterodox physicist E. T. Jaynes \citep{jaynes_site},
which changed the direction of my PhD project and my life.

One day I came across a particular Jaynes paper, which is probably one of my
top five favourite journal articles of all time
\citep{jaynes}. As with many of my favourite papers, I didn't understand it
immediately, but had a strong sense that I should persist because it seemed
important and profound. Every so often, I'd return to re-read it, understanding
just a little more each time. Then, the breakthrough came after dinner one
night.

I had taken out a tub of ice-cream from the freezer for dessert. After dishing
the ice-cream into a bowl, I tried putting the tub back in the freezer, but
it wouldn't fit. Suddenly it clicked: the second law of thermodynamics is the
principle that {\em big things cannot fit into small spaces unless they are
compressed}. This is common sense when thinking of physical objects, but to
get the second law, you have to apply it to an abstract object: a volume
of possibilities.
%A similar, related question is that of how natural selection
%puts information into the genome. Reasonable responses include those by
%\citep{dawkins2} and \citep{mackay}.

%The second law has also been invoked in arguments about the
%sustainability of industrial civilization \citep{muse}. Attenborough






\section*{Entropy increases because it stays constant}

\section*{Entropy: it's not what's happening, it's what you know}



I must admit that when I hear `entropy' from a physicist, without a long
set of qualifications, I don't know what they're talking about.
I don't think this is my fault.

There seems to be little recognition of the fact
that the {\em exact same physical system}
can have more than one entropy, depending on how we want to think about it.
What {\em rules of thumb} are we looking for? Different ones may exist, and
we'll find them by using different entropies.

\begin{thebibliography}{999} % if there are less than 10 entries, enter a one digit number
\bibitem[]{carroll}
Carroll, Sean. From eternity to here: the quest for the ultimate theory of time. Penguin, 2010.

\bibitem[]{dawkins}
Dawkins, Richard. The blind watchmaker: Why the evidence of evolution reveals a universe without design. WW Norton \& Company, 1986.

\bibitem[]{dawkins2}
Dawkins, Richard. The information challenge.
http://www.skeptics.com.au/resources/articles/the-information-challenge/

\bibitem[]{jaynes}
Jaynes, Edwin T.
Gibbs vs Boltzmann entropies. American Journal of Physics 33.5 (1965): 391-398.

\bibitem[]{jaynes_site}
http://bayes.wustl.edu/

\bibitem[]{mackay}
David MacKay, Information Theory, Inference, and Learning Algorithms.
Chapter 19.
http://www.inference.phy.cam.ac.uk/mackay/itprnn/ps/265.280.pdf

\bibitem[]{muse}
Muse. The 2nd law.
http://www.allmusic.com/album/the-2nd-law-mw0002406726
\end{thebibliography}

\end{document}

